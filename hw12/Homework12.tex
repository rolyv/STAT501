
% Default to the notebook output style

    


% Inherit from the specified cell style.




    
\documentclass{article}
    
    
    \usepackage{graphicx} % Used to insert images
    \usepackage{adjustbox} % Used to constrain images to a maximum size 
    \usepackage{color} % Allow colors to be defined
    \usepackage{enumerate} % Needed for markdown enumerations to work
    \usepackage{geometry} % Used to adjust the document margins
    \usepackage{amsmath} % Equations
    \usepackage{amssymb} % Equations
    \usepackage{eurosym} % defines \euro
    \usepackage[mathletters]{ucs} % Extended unicode (utf-8) support
    \usepackage[utf8x]{inputenc} % Allow utf-8 characters in the tex document
    \usepackage{fancyvrb} % verbatim replacement that allows latex
    \usepackage{grffile} % extends the file name processing of package graphics 
                         % to support a larger range 
    % The hyperref package gives us a pdf with properly built
    % internal navigation ('pdf bookmarks' for the table of contents,
    % internal cross-reference links, web links for URLs, etc.)
    \usepackage{hyperref}
    \usepackage{longtable} % longtable support required by pandoc >1.10
    \usepackage{booktabs}  % table support for pandoc > 1.12.2
    \usepackage{indentfirst}
    \usepackage{floatrow}
    
    \definecolor{orange}{cmyk}{0,0.4,0.8,0.2}
    \definecolor{darkorange}{rgb}{.71,0.21,0.01}
    \definecolor{darkgreen}{rgb}{.12,.54,.11}
    \definecolor{myteal}{rgb}{.26, .44, .56}
    \definecolor{gray}{gray}{0.45}
    \definecolor{lightgray}{gray}{.95}
    \definecolor{mediumgray}{gray}{.8}
    \definecolor{inputbackground}{rgb}{.95, .95, .85}
    \definecolor{outputbackground}{rgb}{.95, .95, .95}
    \definecolor{traceback}{rgb}{1, .95, .95}
    % ansi colors
    \definecolor{red}{rgb}{.6,0,0}
    \definecolor{green}{rgb}{0,.65,0}
    \definecolor{brown}{rgb}{0.6,0.6,0}
    \definecolor{blue}{rgb}{0,.145,.698}
    \definecolor{purple}{rgb}{.698,.145,.698}
    \definecolor{cyan}{rgb}{0,.698,.698}
    \definecolor{lightgray}{gray}{0.5}
    
    % bright ansi colors
    \definecolor{darkgray}{gray}{0.25}
    \definecolor{lightred}{rgb}{1.0,0.39,0.28}
    \definecolor{lightgreen}{rgb}{0.48,0.99,0.0}
    \definecolor{lightblue}{rgb}{0.53,0.81,0.92}
    \definecolor{lightpurple}{rgb}{0.87,0.63,0.87}
    \definecolor{lightcyan}{rgb}{0.5,1.0,0.83}
    
    % commands and environments needed by pandoc snippets
    % extracted from the output of `pandoc -s`
    \providecommand{\tightlist}{%
      \setlength{\itemsep}{0pt}\setlength{\parskip}{0pt}}
    \DefineVerbatimEnvironment{Highlighting}{Verbatim}{commandchars=\\\{\}}
    % Add ',fontsize=\small' for more characters per line
    \newenvironment{Shaded}{}{}
    \newcommand{\KeywordTok}[1]{\textcolor[rgb]{0.00,0.44,0.13}{\textbf{{#1}}}}
    \newcommand{\DataTypeTok}[1]{\textcolor[rgb]{0.56,0.13,0.00}{{#1}}}
    \newcommand{\DecValTok}[1]{\textcolor[rgb]{0.25,0.63,0.44}{{#1}}}
    \newcommand{\BaseNTok}[1]{\textcolor[rgb]{0.25,0.63,0.44}{{#1}}}
    \newcommand{\FloatTok}[1]{\textcolor[rgb]{0.25,0.63,0.44}{{#1}}}
    \newcommand{\CharTok}[1]{\textcolor[rgb]{0.25,0.44,0.63}{{#1}}}
    \newcommand{\StringTok}[1]{\textcolor[rgb]{0.25,0.44,0.63}{{#1}}}
    \newcommand{\CommentTok}[1]{\textcolor[rgb]{0.38,0.63,0.69}{\textit{{#1}}}}
    \newcommand{\OtherTok}[1]{\textcolor[rgb]{0.00,0.44,0.13}{{#1}}}
    \newcommand{\AlertTok}[1]{\textcolor[rgb]{1.00,0.00,0.00}{\textbf{{#1}}}}
    \newcommand{\FunctionTok}[1]{\textcolor[rgb]{0.02,0.16,0.49}{{#1}}}
    \newcommand{\RegionMarkerTok}[1]{{#1}}
    \newcommand{\ErrorTok}[1]{\textcolor[rgb]{1.00,0.00,0.00}{\textbf{{#1}}}}
    \newcommand{\NormalTok}[1]{{#1}}
    
    % Define a nice break command that doesn't care if a line doesn't already
    % exist.
    \def\br{\hspace*{\fill} \\* }
    % Math Jax compatability definitions
    \def\gt{>}
    \def\lt{<}
    % Document parameters
    \title{Homework 12}
    \author{Roly Vicar\'ia \\ STAT501 Fall 2015}    
    
    

    % Pygments definitions
    
\makeatletter
\def\PY@reset{\let\PY@it=\relax \let\PY@bf=\relax%
    \let\PY@ul=\relax \let\PY@tc=\relax%
    \let\PY@bc=\relax \let\PY@ff=\relax}
\def\PY@tok#1{\csname PY@tok@#1\endcsname}
\def\PY@toks#1+{\ifx\relax#1\empty\else%
    \PY@tok{#1}\expandafter\PY@toks\fi}
\def\PY@do#1{\PY@bc{\PY@tc{\PY@ul{%
    \PY@it{\PY@bf{\PY@ff{#1}}}}}}}
\def\PY#1#2{\PY@reset\PY@toks#1+\relax+\PY@do{#2}}

\expandafter\def\csname PY@tok@gd\endcsname{\def\PY@tc##1{\textcolor[rgb]{0.63,0.00,0.00}{##1}}}
\expandafter\def\csname PY@tok@gu\endcsname{\let\PY@bf=\textbf\def\PY@tc##1{\textcolor[rgb]{0.50,0.00,0.50}{##1}}}
\expandafter\def\csname PY@tok@gt\endcsname{\def\PY@tc##1{\textcolor[rgb]{0.00,0.27,0.87}{##1}}}
\expandafter\def\csname PY@tok@gs\endcsname{\let\PY@bf=\textbf}
\expandafter\def\csname PY@tok@gr\endcsname{\def\PY@tc##1{\textcolor[rgb]{1.00,0.00,0.00}{##1}}}
\expandafter\def\csname PY@tok@cm\endcsname{\let\PY@it=\textit\def\PY@tc##1{\textcolor[rgb]{0.25,0.50,0.50}{##1}}}
\expandafter\def\csname PY@tok@vg\endcsname{\def\PY@tc##1{\textcolor[rgb]{0.10,0.09,0.49}{##1}}}
\expandafter\def\csname PY@tok@m\endcsname{\def\PY@tc##1{\textcolor[rgb]{0.40,0.40,0.40}{##1}}}
\expandafter\def\csname PY@tok@mh\endcsname{\def\PY@tc##1{\textcolor[rgb]{0.40,0.40,0.40}{##1}}}
\expandafter\def\csname PY@tok@go\endcsname{\def\PY@tc##1{\textcolor[rgb]{0.53,0.53,0.53}{##1}}}
\expandafter\def\csname PY@tok@ge\endcsname{\let\PY@it=\textit}
\expandafter\def\csname PY@tok@vc\endcsname{\def\PY@tc##1{\textcolor[rgb]{0.10,0.09,0.49}{##1}}}
\expandafter\def\csname PY@tok@il\endcsname{\def\PY@tc##1{\textcolor[rgb]{0.40,0.40,0.40}{##1}}}
\expandafter\def\csname PY@tok@cs\endcsname{\let\PY@it=\textit\def\PY@tc##1{\textcolor[rgb]{0.25,0.50,0.50}{##1}}}
\expandafter\def\csname PY@tok@cp\endcsname{\def\PY@tc##1{\textcolor[rgb]{0.74,0.48,0.00}{##1}}}
\expandafter\def\csname PY@tok@gi\endcsname{\def\PY@tc##1{\textcolor[rgb]{0.00,0.63,0.00}{##1}}}
\expandafter\def\csname PY@tok@gh\endcsname{\let\PY@bf=\textbf\def\PY@tc##1{\textcolor[rgb]{0.00,0.00,0.50}{##1}}}
\expandafter\def\csname PY@tok@ni\endcsname{\let\PY@bf=\textbf\def\PY@tc##1{\textcolor[rgb]{0.60,0.60,0.60}{##1}}}
\expandafter\def\csname PY@tok@nl\endcsname{\def\PY@tc##1{\textcolor[rgb]{0.63,0.63,0.00}{##1}}}
\expandafter\def\csname PY@tok@nn\endcsname{\let\PY@bf=\textbf\def\PY@tc##1{\textcolor[rgb]{0.00,0.00,1.00}{##1}}}
\expandafter\def\csname PY@tok@no\endcsname{\def\PY@tc##1{\textcolor[rgb]{0.53,0.00,0.00}{##1}}}
\expandafter\def\csname PY@tok@na\endcsname{\def\PY@tc##1{\textcolor[rgb]{0.49,0.56,0.16}{##1}}}
\expandafter\def\csname PY@tok@nb\endcsname{\def\PY@tc##1{\textcolor[rgb]{0.00,0.50,0.00}{##1}}}
\expandafter\def\csname PY@tok@nc\endcsname{\let\PY@bf=\textbf\def\PY@tc##1{\textcolor[rgb]{0.00,0.00,1.00}{##1}}}
\expandafter\def\csname PY@tok@nd\endcsname{\def\PY@tc##1{\textcolor[rgb]{0.67,0.13,1.00}{##1}}}
\expandafter\def\csname PY@tok@ne\endcsname{\let\PY@bf=\textbf\def\PY@tc##1{\textcolor[rgb]{0.82,0.25,0.23}{##1}}}
\expandafter\def\csname PY@tok@nf\endcsname{\def\PY@tc##1{\textcolor[rgb]{0.00,0.00,1.00}{##1}}}
\expandafter\def\csname PY@tok@si\endcsname{\let\PY@bf=\textbf\def\PY@tc##1{\textcolor[rgb]{0.73,0.40,0.53}{##1}}}
\expandafter\def\csname PY@tok@s2\endcsname{\def\PY@tc##1{\textcolor[rgb]{0.73,0.13,0.13}{##1}}}
\expandafter\def\csname PY@tok@vi\endcsname{\def\PY@tc##1{\textcolor[rgb]{0.10,0.09,0.49}{##1}}}
\expandafter\def\csname PY@tok@nt\endcsname{\let\PY@bf=\textbf\def\PY@tc##1{\textcolor[rgb]{0.00,0.50,0.00}{##1}}}
\expandafter\def\csname PY@tok@nv\endcsname{\def\PY@tc##1{\textcolor[rgb]{0.10,0.09,0.49}{##1}}}
\expandafter\def\csname PY@tok@s1\endcsname{\def\PY@tc##1{\textcolor[rgb]{0.73,0.13,0.13}{##1}}}
\expandafter\def\csname PY@tok@kd\endcsname{\let\PY@bf=\textbf\def\PY@tc##1{\textcolor[rgb]{0.00,0.50,0.00}{##1}}}
\expandafter\def\csname PY@tok@sh\endcsname{\def\PY@tc##1{\textcolor[rgb]{0.73,0.13,0.13}{##1}}}
\expandafter\def\csname PY@tok@sc\endcsname{\def\PY@tc##1{\textcolor[rgb]{0.73,0.13,0.13}{##1}}}
\expandafter\def\csname PY@tok@sx\endcsname{\def\PY@tc##1{\textcolor[rgb]{0.00,0.50,0.00}{##1}}}
\expandafter\def\csname PY@tok@bp\endcsname{\def\PY@tc##1{\textcolor[rgb]{0.00,0.50,0.00}{##1}}}
\expandafter\def\csname PY@tok@c1\endcsname{\let\PY@it=\textit\def\PY@tc##1{\textcolor[rgb]{0.25,0.50,0.50}{##1}}}
\expandafter\def\csname PY@tok@kc\endcsname{\let\PY@bf=\textbf\def\PY@tc##1{\textcolor[rgb]{0.00,0.50,0.00}{##1}}}
\expandafter\def\csname PY@tok@c\endcsname{\let\PY@it=\textit\def\PY@tc##1{\textcolor[rgb]{0.25,0.50,0.50}{##1}}}
\expandafter\def\csname PY@tok@mf\endcsname{\def\PY@tc##1{\textcolor[rgb]{0.40,0.40,0.40}{##1}}}
\expandafter\def\csname PY@tok@err\endcsname{\def\PY@bc##1{\setlength{\fboxsep}{0pt}\fcolorbox[rgb]{1.00,0.00,0.00}{1,1,1}{\strut ##1}}}
\expandafter\def\csname PY@tok@mb\endcsname{\def\PY@tc##1{\textcolor[rgb]{0.40,0.40,0.40}{##1}}}
\expandafter\def\csname PY@tok@ss\endcsname{\def\PY@tc##1{\textcolor[rgb]{0.10,0.09,0.49}{##1}}}
\expandafter\def\csname PY@tok@sr\endcsname{\def\PY@tc##1{\textcolor[rgb]{0.73,0.40,0.53}{##1}}}
\expandafter\def\csname PY@tok@mo\endcsname{\def\PY@tc##1{\textcolor[rgb]{0.40,0.40,0.40}{##1}}}
\expandafter\def\csname PY@tok@kn\endcsname{\let\PY@bf=\textbf\def\PY@tc##1{\textcolor[rgb]{0.00,0.50,0.00}{##1}}}
\expandafter\def\csname PY@tok@mi\endcsname{\def\PY@tc##1{\textcolor[rgb]{0.40,0.40,0.40}{##1}}}
\expandafter\def\csname PY@tok@gp\endcsname{\let\PY@bf=\textbf\def\PY@tc##1{\textcolor[rgb]{0.00,0.00,0.50}{##1}}}
\expandafter\def\csname PY@tok@o\endcsname{\def\PY@tc##1{\textcolor[rgb]{0.40,0.40,0.40}{##1}}}
\expandafter\def\csname PY@tok@kr\endcsname{\let\PY@bf=\textbf\def\PY@tc##1{\textcolor[rgb]{0.00,0.50,0.00}{##1}}}
\expandafter\def\csname PY@tok@s\endcsname{\def\PY@tc##1{\textcolor[rgb]{0.73,0.13,0.13}{##1}}}
\expandafter\def\csname PY@tok@kp\endcsname{\def\PY@tc##1{\textcolor[rgb]{0.00,0.50,0.00}{##1}}}
\expandafter\def\csname PY@tok@w\endcsname{\def\PY@tc##1{\textcolor[rgb]{0.73,0.73,0.73}{##1}}}
\expandafter\def\csname PY@tok@kt\endcsname{\def\PY@tc##1{\textcolor[rgb]{0.69,0.00,0.25}{##1}}}
\expandafter\def\csname PY@tok@ow\endcsname{\let\PY@bf=\textbf\def\PY@tc##1{\textcolor[rgb]{0.67,0.13,1.00}{##1}}}
\expandafter\def\csname PY@tok@sb\endcsname{\def\PY@tc##1{\textcolor[rgb]{0.73,0.13,0.13}{##1}}}
\expandafter\def\csname PY@tok@k\endcsname{\let\PY@bf=\textbf\def\PY@tc##1{\textcolor[rgb]{0.00,0.50,0.00}{##1}}}
\expandafter\def\csname PY@tok@se\endcsname{\let\PY@bf=\textbf\def\PY@tc##1{\textcolor[rgb]{0.73,0.40,0.13}{##1}}}
\expandafter\def\csname PY@tok@sd\endcsname{\let\PY@it=\textit\def\PY@tc##1{\textcolor[rgb]{0.73,0.13,0.13}{##1}}}

\def\PYZbs{\char`\\}
\def\PYZus{\char`\_}
\def\PYZob{\char`\{}
\def\PYZcb{\char`\}}
\def\PYZca{\char`\^}
\def\PYZam{\char`\&}
\def\PYZlt{\char`\<}
\def\PYZgt{\char`\>}
\def\PYZsh{\char`\#}
\def\PYZpc{\char`\%}
\def\PYZdl{\char`\$}
\def\PYZhy{\char`\-}
\def\PYZsq{\char`\'}
\def\PYZdq{\char`\"}
\def\PYZti{\char`\~}
% for compatibility with earlier versions
\def\PYZat{@}
\def\PYZlb{[}
\def\PYZrb{]}
\makeatother


    % Exact colors from NB
    \definecolor{incolor}{rgb}{0.0, 0.0, 0.5}
    \definecolor{outcolor}{rgb}{0.545, 0.0, 0.0}



    
    % Prevent overflowing lines due to hard-to-break entities
    \sloppy 
    % Setup hyperref package
    \hypersetup{
      breaklinks=true,  % so long urls are correctly broken across lines
      colorlinks=true,
      urlcolor=blue,
      linkcolor=darkorange,
      citecolor=darkgreen,
      }
    % Slightly bigger margins than the latex defaults
    
    \geometry{verbose,tmargin=1in,bmargin=1in,lmargin=1in,rmargin=1in}
    
    

    \begin{document}
    
    
    \maketitle
    
    

    
    \subsubsection{Question 1}\label{question-1}

\begin{enumerate}
\def\labelenumi{\alph{enumi})}
\tightlist
\item
  There appears to be a slight upward trend in the data.
  
  \begin{figure}[h!]
 \centering
 \includegraphics[scale=.45]{./images/time-series-plot_NovLevel.png}
 % time-series-plot_NovLevel.png: 575x384 pixel, 96dpi, 15.21x10.16 cm, bb=0 0 431 288
\end{figure}

\end{enumerate}

\begin{enumerate}
\def\labelenumi{\alph{enumi})}
\setcounter{enumi}{1}
\tightlist
\item
  The plot of NovLevel vs the Lag1level does show a positive, linear
  pattern suggesting that the first-order autogression model could be
  useful.
  
  \begin{figure}[h!]
 \centering
 \includegraphics[scale=.45]{./images/scatterplot_NovLevel-vs-Lag1Level.png}
 % scatterplot_NovLevel-vs-Lag1Level.png: 575x383 pixel, 96dpi, 15.21x10.13 cm, bb=0 0 431 287
\end{figure}

\end{enumerate}

\newpage
\begin{enumerate}
\def\labelenumi{\alph{enumi})}
\setcounter{enumi}{2}
\tightlist
\item
  The partial autocorrelation shows a significant spike at a lag of 1
  and much lower spikes for the subsequent lags. It indicates that a
  AR(1) model would likely be feasible for this data set.
  
  \begin{figure}[h!]
 \centering
 \includegraphics[scale=.45]{./images/PACF_novLevel.png}
 % PACF_novLevel.png: 575x383 pixel, 96dpi, 15.21x10.13 cm, bb=0 0 431 287
\end{figure}

\end{enumerate}

\begin{enumerate}
\def\labelenumi{\alph{enumi})}
\setcounter{enumi}{3}
\item
  \begin{enumerate}
  \def\labelenumii{\roman{enumii}.}
  \tightlist
  \item
    NovLevel = 3.88 +~0.723~Lag1level
  \item
    14.5841
  \end{enumerate}
\item
  The partial autocorrelation shows only weak correlations, so the
  residuals could reasonably be assumed to be independent of each other.
  
  \begin{figure}[h!]
 \centering
 \includegraphics[scale=.45]{./images/PACF_residuals.png}
 % PACF_residuals.png: 575x383 pixel, 96dpi, 15.21x10.13 cm, bb=0 0 431 287
\end{figure}

\end{enumerate}

    \subsubsection{Question 2}\label{question-2}

\begin{enumerate}
\def\labelenumi{\alph{enumi})}
\item
  \(\hat{p} = \dfrac{e^{-0.573 - 0.498(GPA)}}{1 + e^{-0.573 - 0.498(GPA)}}\)
\item
  The plot shows that the probability of cheating on an exam decreases
  as the GPA increases.
  
  \begin{figure}[h!]
 \centering
 \includegraphics[scale=.45]{./images/scatterplot_fits-vs-GPA.png}
 % scatterplot_fits-vs-GPA.png: 575x383 pixel, 96dpi, 15.21x10.13 cm, bb=0 0 431 287
\end{figure}

\end{enumerate}

\begin{enumerate}
\def\labelenumi{\alph{enumi})}
\setcounter{enumi}{2}
\item
  The odds ratio for GPA is 0.6076. This can be interpreted as saying
  that for every 1 point increase in GPA, the odds of having cheated on
  an exam are multiplied by 0.6076.
\item
  \(\hat{p} = \dfrac{e^{-0.573 - 0.498(3.0)}}{1 + e^{-0.573 - 0.498(3.0)}} = 0.1266\)
\item
  \(\hat{p} = \dfrac{e^{-0.573 - 0.498(4.0)}}{1 + e^{-0.573 - 0.498(4.0)}} = 0.07692\)
\end{enumerate}

\(Odds = \dfrac{p}{1 - p} = \dfrac{0.07692}{1 - 0.07692} = 0.0833\)

\begin{enumerate}
\def\labelenumi{\alph{enumi})}
\setcounter{enumi}{5}
\item
  In the expanded tables output, we can see the Z-value and p-value for
  the SkipClass term which shows a p-value of 0.001 indicating a
  significant predictor.
\item
  The odds ratio for SkipClass is 1.4138. This can be interpreted as
  saying that for every additional class skipped by a student in a
  typical week, the odds of having cheated on an exam are multiplied by
  1.4138, assuming GPA is held constant.
\item
  \(\dfrac{p}{1-p} = e^{-1.190 - 0.375(3.5) + 0.346(1)} = 0.1157\)
\item
  0.1157 * 1.4138 = 0.1636
\item
  Of the 4 additional categorical predictors, it looks like SmokeCig and
  FakeID are significant predictors of ever having cheated on an exam
  with p-values of .005 and .010, respectively. ChtdSO is on the fence
  since it's p-value is less than 0.1, but I will not consider it
  significant here. The 4th predictor, SmokeMJ, has a very high p-value
  of .412 and therefore not a significant predictor of having cheated on
  an exam.

The odds ratio of SmokeCig is 2.2769 which says that the odds of ever
having cheated on an exam for students who smoke cigarettes is 2.2769
times higher than that of students who do not smoke cigarettes (all else
being constant).

The odds ratio of FakeID is 1.9454 which says that the odds of ever
having cheated on an exam for students who have ever used a fake id to
obtain alcohol is 1.9454 times higher than that of students who have
never used a fake id (all else being constant).
\end{enumerate}

    \subsubsection{Question 3}\label{question-3}

\begin{enumerate}
\def\labelenumi{\alph{enumi})}
\item
  \(Falls = e^{0.489 - 1.069(Inter) - 0.047(Gender) + 0.00947(Balance) + 0.00857(Strength)}\)
\end{enumerate}

\begin{enumerate}
 \def\labelenumi{\alph{enumi})}
 \setcounter{enumi}{1}
 \item 
\(H_0: \beta_{Gender} = 0\)  \\
\(H_a: \beta_{Gender} \ne 0\) \\
Test statistic: \(G^2 = 108.941 - 108.790 = 0.151\) \\
p-value = 0.698

Based on the p-value, we do not have enough evidence to reject the null
hypothesis, therefore we conclude that \(\beta_{Gender} = 0\).
\end{enumerate}

\begin{enumerate}
\def\labelenumi{\alph{enumi})}
\setcounter{enumi}{2}
\tightlist
\item
  The Residuals vs order plot does not reveal any outlying cases, values
  are generally in the -2 to 2 range.
  
  \begin{figure}[h!]
 \centering
 \includegraphics[scale=.4]{./images/plot_residuals-vs-order.png}
 % plot_residuals-vs-order.png: 575x383 pixel, 96dpi, 15.21x10.13 cm, bb=0 0 431 287
\end{figure}

\end{enumerate}

\newpage
\begin{enumerate}
\def\labelenumi{\alph{enumi})}
\setcounter{enumi}{3}
\tightlist
\item
  \(Falls = e^{0.444 - 1.078(Inter) + 0.00947(Balance) + 0.00898(Strength)}\)
  
  
Based on this estimate equation, aerobic exercise does reduce the
frequency of falls when controlling for balance and strength.
\end{enumerate}


\begin{enumerate}
\def\labelenumi{\alph{enumi})}
\setcounter{enumi}{4}
\tightlist
\item
  Based on the Pearson and Deviance statistics' p-values, 0.243 and
  0.173, respectively, there is no evidence of lack-of-fit.

  \(R^2 = 1 - \dfrac{-2\ell(\hat{\beta})}{-2\ell(\hat{\beta_0}} = 1 - \dfrac{108.941}{199.194} = 0.4531\)
  \end{enumerate}


    % Add a bibliography block to the postdoc
    
    
    
    \end{document}
